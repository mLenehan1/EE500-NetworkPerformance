\documentclass[a4paper]{article}

\usepackage[utf8]{inputenc}
\usepackage[T1]{fontenc}
\usepackage{textcomp}
\usepackage[dutch]{babel}
\usepackage{amsmath, amssymb}


% figure support
\usepackage{import}
\usepackage{xifthen}
\pdfminorversion=7
\usepackage{pdfpages}
\usepackage{transparent}
\newcommand{\incfig}[1]{%
	\def\svgwidth{\columnwidth}
	\import{./figures/}{#1.pdf_tex}
}

\pdfsuppresswarningpagegroup=1

\begin{document}
\section{Performance at Transport Layer}
\subsection{TCP}
Characteristics
\begin{itemize}
	\item Connection-oriented reliable byte-stream service
	\begin{itemize}
		\item Two comms end points must establish, maintain connection
		\item Byte stream data exchanged between end points
		\item Reliable data delivery ensured
	\end{itemize}
\end{itemize}
\subsubsection{Principles}
\begin{itemize}
	\item Data broken into segments of certain size
	\begin{itemize}
		\item Max. Segment Size (MSS) set (negotiated) at conn. estab.
		\item MSS val carried by SYN segment
		\item End point never TX segment larger than MSS
		\item Higher MSS, better performance (min. OH vs payload ratio)
		\item If not indicated, default val of 536 bytes is used for MSS
		\item In reality, seg size 40 byteslarger as also includes
			\begin{itemize}
				\item 20 bytes TCO header
				\item 20 bytes IP header
			\end{itemize}
	\end{itemize}
\end{itemize}
\subsubsection{Problem}
\begin{itemize}
	\item Data broken into segments of certain size
	\begin{itemize}
		\item COmpute efficiency considering OH and payload when TX 4GB
			video and MSS of 1460and 536 used:
		\begin{itemize}
			\item Case 1: No pkts: $4*\frac{1024}{1460}=2873;
				Headers: 40*2873=114920$
			\begin{itemize}
				\item Efficiency = useful data/total data =
					$\frac{4194304}{4309224}=97.33\%$
			\end{itemize}
			\item Case 2: No pkts: $4*\frac{1024}{536}=7826;
				Headers: 40*7826=313040$
			\begin{itemize}
				\item Efficiency=useful data/total data=
					$\frac{419304}{4507344}=93.05\% $
			\end{itemize}
		\end{itemize}
	\end{itemize}
\end{itemize}
\subsubsection{Principles}
\begin{itemize}
	\item Reliability through ack and reTX
	\item When TX seg, timer set
	\item RX expected to ack seg
	\item If ack not RX, seg is reTX
\end{itemize}
\subsubsection{Issues}
\begin{itemize}
	\item ReTX TimeOut (RTO) period is variable
	\begin{itemize}
		\item RTO set in relation to RTT
		\item RTT estimated using smoothed estimator (SRTT) using a
			low-pass filter
		\item After unsuccessful reTX, TO period doubled (exponential
			backoff) with an upper limit
	\end{itemize}
	\item Adter series of unsuccessful reTX, conn. is reset
	\begin{itemize}
		\item TCP implementation have Keepalive timer, not present in
			TCP standard
	\end{itemize}
\end{itemize}
\subsubsection{Problem}
\begin{itemize}
	\item Exponential backoff reTX
	\begin{enumerate}
		\item List times at which reTX occurs if there is a need for 6
			reTX and the first two are 2s apart
		\begin{itemize}
			\item ReTX: 1,3,7,15,31,63
		\end{itemize}
		\item List times at which reTX occur if there is need for 10
			reTX and the first two are 1.5s apart
		\begin{itemize}
			\item ReTX:
				1,2.5,6.5,11.5,23.5,47.5,96.5,159.5,223.5,287.5
		\end{itemize}
	\end{enumerate}
\end{itemize}
\subsubsection{RTT Estimation}
\begin{itemize}
	\item Calculated every time new measurement performed
	\item $STT = \alpha * SRTT+(1-\alpha)*MRTT$, where
	\begin{itemize}
		\item SRTT is RTT estimator
		\item $\alpha$ is smoothing factor with rec. val between 0.8 and
			0.9
		\item MRTT is measured RTT
	\end{itemize}
\end{itemize}
\subsubsection{RTO Value}
\begin{itemize}
	\item Calculated every time there is need for reTX
	\item $RTO = min (RTOMax, max(RTOMin,(\beta * SRTT)))$, where
	 \begin{itemize}
		\item RTOMax is upper bound on timeout (e.g. 1m, 64sec)
		\item RTOMin is lower bound on timeout (e.g. 1s)
		\item $\beta$ is delay variance factor with dixed val between
			1.3 and 2
		\item If ack not RX, seg is reTX
	\end{itemize}
\end{itemize}
\subsubsection{ISsues}
\begin{itemize}
	\item RTT estimation accuracy problem
	\begin{itemize}
		\item RTT estimation alg assumes RTT variations are small,
			constant
		\item Loses accuracy with wide fluctuations in RTT, causing
			unnecessary reTX
		\item ReTX add traffic to already loaded net.
	\end{itemize}
	\item Jacobson's soln.
	\begin{itemize}
		\item Keep track of both mean and variance of RTT, compute RTO
			based on both
	\end{itemize}
\end{itemize}
\subsubsection{RTT Average and Mean Deviation}
\begin{itemize}
	\item Calc every time new measurement performed
	\item $Err = MRTT - ARTT$
	\begin{itemize}
		\item Err is error between measured val and smothered value for
			RTT
		\item ARTT is smothered RTT average
		\item g is gain factor with rec. value of 1/8
		\item MRTT is measured RTT
	\end{itemize}
\end{itemize}
\subsubsection{RTT MEan Deviation}
\begin{itemize}
	\item Calc. every time new measurement performed
	\item $DRTT = DRTT*h*(|Err|-DRTT)$
	\begin{itemize}
		\item DRTT is smothered mean deviation
		\item h is difference factor set to 0.25
	\end{itemize}
\end{itemize}
\subsubsection{RTO value}
\begin{itemize}
	\item Calc. every time need for reTX
	\item $RTO = ARTT+r*DRTT$, where
	\begin{itemize}
		\item r is constant set to 4
		\item Initial val set for r was 2, later changed to 4
	\end{itemize}
\end{itemize}
\subsubsection{Issues}
\begin{itemize}
	\item RTT measurement accuracy problem
	\begin{itemize}
	 	\item RTT measured between sending of data pkt and its ack
		\item When Large delays occur, timer goes off, reTX takes place
		\item When RX ack, no way to know if it was delayed res. to orig data
			seg or res. to reTX seg
	\end{itemize}
	\item Karn's soln
	\begin{itemize}
		\item Not to update RTT estimator with info on reTX seg's
			performance
	\end{itemize}
	\item Limitations of Karn;s soln
	\begin{itemize}
		\item When RTT increases sharply, normal reTX resumed after
			series of reTX and TCP does to receive ack, for a while
			RTT not updated, reTX would happen considering old RTT
			val
	\end{itemize}
	\item Limitations soln
	\begin{itemize}
		\item Exponential backoff timer timout val employed:
			timout=2*timout
	\end{itemize}
	\item Solution isse
		\begin{itemize}
			\item If to increases too much, delays added with no
				correlation with actual net. delay
		\end{itemize}
	\item Solution fix
	\begin{itemize}
		\item Upper limitations added to to value > 1 min: 64s
	\end{itemize}
\end{itemize}
\subsubsection{Computation of RTT estimation}
\subsection{Congestion control}
\begin{itemize}
	\item Modern TCP std include 4 major alg
	\begin{itemize}
		\item Slow start
		\item COngestion avoidance
		\item Fast reTX
		\item Fast recovery
	\end{itemize}
\end{itemize}
\subsection{Slow Start}
\subsubsection{Principle}
\begin{itemize}
	\item Slowly probes net in order to determine available capacity
	\item Employed at beginning of transfer/after loss detected by TX timer
	\item Uses following var.
		\begin{itemize}
			\item COngestion window (cwnd) - sender side limit on
				amount which can be TX before receiving ack
			\item Receiver window (rwnd) - Receiver-side limit on
				outstanding data
			\item Slow start threshold (ssthresh) - limit to decide
				using slow start or congestion avoidance
			\item Sender Maximum Segment Size (SMSS) - Max seg size
				at sender
			\item Flight Size - amount of unack data in TX between
				sender, receiver
		\end{itemize}
	\item TX should exchange min of cwnd and rwnd amount of data
	\item Slow start used when cwnd > ssthresh, Congestion Avoidance
		emploued for cwnd < ssthresh and either alg when cwnd = ssthresh
\end{itemize}
\subsubsection{MEchanism}
\begin{itemize}
	\item
	\item
	\item
	\item
	\item
\end{itemize}
\subsubsection{Note}
\begin{itemize}
	\item
\end{itemize}
\subsubsection{Problem: ACK Division}
\begin{itemize}
	\item `
	\item
	\item
\end{itemize}
\subsubsection{UPdated Mechanism}
\begin{itemize}
	\item
	\item
	\item
\end{itemize}
\subsubsection{Performance Issues}
\begin{itemize}
	\item
	\begin{itemize}
		\item
		\begin{itemize}
			\item
		\end{itemize}
	\end{itemize}
	\item
	\begin{itemize}
		\item
		\begin{itemize}
			\item
		\end{itemize}
		\item
		\begin{itemize}
			\item
		\end{itemize}
	\end{itemize}
	\item
	\begin{itemize}
		\item
		\begin{itemize}
			\item
		\end{itemize}
		\item
		\begin{itemize}
			\item
		\end{itemize}
		\item
		\begin{itemize}
			\item
			\item
			\item
			\item
			\item
		\end{itemize}
	\end{itemize}
\end{itemize}
\subsection{Congestion Avoidance}
\subsubsection{Principle}
\subsubsection{Mechanism}
\subsubsection{Updated MEchanism}
\subsubsection{Note}
\subsection{Fast Retransmit}
\subsubsection{Principle}
\subsubsection{Mechanism}
\subsection{Fast Recovery}
\subsubsection{Principle}
\subsubsection{Mechanism}
\subsubsection{Note}
\subsection{Fast Retransmit and Fast Recovery}
\subsubsection{PRinciple}
\begin{itemize}
	\item Two major alg types that improve Fast reTX and Fast recovery
	\item Based on TCP selective Ack
	\begin{itemize}
		\item
	\end{itemize}

\end{itemize}
\section{TCP Tahoe}
\subsection{Characteristics}
\begin{itemize}
	\item Fast recovery not included
	\item Fast reTX not included
	\item TXP old tahoe did not have fast reTX either
	\item Implemented in Unix 4.3 BSD Tahoe
\end{itemize}
\subsection{ISsues}
\begin{itemize}
	\item ONly mechanism to detect loss is through reTX timer timeout
	\begin{itemize}
		\item Introduces potential delays
	\end{itemize}
	\item TCP old tahoe by not emplying dast reTX alg, slow start has to be
		used
	\begin{itemize}
		\item Rates kept low
	\end{itemize}
	\item Every lost pkt determines cwnd reste to min
	\begin{itemize}
		\item Severe reduction in rates
	\end{itemize}
\end{itemize}
\section{TCP Reno}
\subsection{Characteristics}
\subsection{Issues}
\section{TCP NEw Reno, SACK and Vegas}
\subsection{TCP New Reno}
\subsubsection{Characteristics}
\subsubsection{ISsues}
\subsection{TCP SACK}
\subsubsection{Characteristics}
\subsubsection{Issues}
\subsection{TCP Vegas}
\subsubsection{Characteristics}
\subsubsection{Issues}
\section{SCTP}
\subsection{Motivation}
\begin{itemize}
	\item TCP limitations with wireless and mobile comms
	\item Need for multi-streaming
	\item Need for multi-homing
\end{itemize}
\subsection{OVerview}
\begin{itemize}
	\item Series of IETF 2960 (2000), IETF RFC 3286 (2002)
\end{itemize}
\subsection{Features}
\begin{itemize}
	\item Reliable transport protocol
	\item Uses association instead of conn.
	\item Designed for message oriented applications
		\begin{itemize}
			\item Framing (preserve message boundaries)
		\end{itemize}
	\item Ack error free transfer of msg
	\item Detection of data corruption, data loss and data duplication
	\item Selective reTX to corect lost or corrupted data
	\item Active monitoring of session conn. via heartbeat
	\item Resistance to DOS attacks
	\begin{itemize}
		\item 4-way handshake
	\end{itemize}
	\item Supports multi-streaming
	\begin{itemize}
		\item Up to 64K indp. ordered streams
	\end{itemize}
	\item Supports multi-homing
	\begin{itemize}
		\item Set of IP addresses per endpoint
	\end{itemize}
\end{itemize}
\subsection{Message Format - 1:HEader}
\begin{itemize}
	\item Src Port and Dest Port (2+2 bytes)
	\begin{itemize}
		\item Same port concept as TCP and UDP
	\end{itemize}
	\item Verification Tag (4 bytes)
	\begin{itemize}
		\item Exchanged etween endpoints at startup to validate the
			sender
	\end{itemize}
\item Checksum (4 bytes)
	\begin{itemize}
		\item Uses CRC32 alg
	\end{itemize}
\end{itemize}
\subsection{MEssage Format - 2: Chuncks}
\begin{itemize}
	\item Type (1 byte)
	\begin{itemize}
		\item Control or Data: e.g. Data, Init, SACK
	\end{itemize}
\item Flags (1 byte)
	\begin{itemize}
		\item Carry info depending on type
	\end{itemize}
\item Length (2 bytes)
	\begin{itemize}
		\item Chunk length, including data payload length
	\end{itemize}
\item Data (N bytes)
	\begin{itemize}
		\item Variable length payload
	\end{itemize}
\end{itemize}
\subsection{Message Format - 3: Important Chunk Types}
\begin{itemize}
	\item DATA
	\begin{itemize}
		\item IDs chunks carrying data
	\end{itemize}
	\item INIT, INIT-ACK, COOKIE-ECHO, COOKIE-ACK
	\begin{itemize}
		\item USedfor association establishment
	\end{itemize}
	\item HEARTBEAT, HEARTBEAT-ACK
	\begin{itemize}
		\item Used for keep-alive chacking
	\end{itemize}
	\item SHUTDOWN, SHUTDOWN-ACK
	\begin{itemize}
		\item USed for graceful disconn.
	\end{itemize}
\end{itemize}
\subsection{Establishing an Association}
Association Establishment Procedure
\begin{itemize}
	\item
\end{itemize}
\subsection{INIT Chunk}
\begin{itemize}
	\item Initiate Tag
	\begin{itemize}
		\item Receiver stores Initiate Tag Value
		\item Must be placed in VErification Tag field of every SCTP pkt
			receiver sends
	\end{itemize}
\item Advertised Receiver Window Credit (a\_rwnd)
	\begin{itemize}
		\item Indicates dedicated buffer space sender reserved for this
			association
	\end{itemize}
\item Number of Outbound Streams (OS)
	\begin{itemize}
		\item Number of outbound streams sender wishes to create in this
			association (max 64k)
	\end{itemize}
\item Number of Inbound Streams (I-TSN)
	\begin{itemize}
		\item Defines initial TX seq number the sender will use
		\item Field may be set to value of Initiate Tag Field
	\end{itemize}
	\item
	\begin{itemize}
		\item
	\end{itemize}
\end{itemize}
\subsection{INIT-ACK Chunk}
\subsection{C\OKIE ECHO and C\OKIE ACK Chunks}
\subsection{DATA Chunk}
\subsection{Terminating an Association}
\subsection{SHUTDOWN Chunks}
\subsection{Multihoming}
SCTP Association:
\begin{itemize}
	\item Comm. hosts use set of IP addr. instead of single one each
	\item Multi comms path may be set up
	\begin{itemize}
		\item One primary path
		\item No. of secondary paths
	\end{itemize}
	\item Lists of IP addrs exchanged between hosts during init of assoc.
	\begin{itemize}
		\item Both INIT and INITACK msgs include list of IP addrs
	\end{itemize}
	\item Source of INIT msg is dest of INIT-ACK
	\begin{itemize}
		\item In general, determine primary path
	\end{itemize}
\end{itemize}
SCTP Operation:
\begin{itemize}
	\item HOsts monitor data TOs and No. of reTX to determine path's
		transmission quality
	\item ReTX Data chunks may be sent over multipaths if status of one path
		is suspect
	\item Faulty paths marked ``Out of Service''
	\item HEartbeat chunks sent periodically to all inactive IP addr
	\item Non-responding IP addrs will be marked ``Out of Service''
\end{itemize}
\section{mSCTP}
Mobile SCTP
\begin{itemize}
	\item Extends SCTP
	\begin{itemize}
		\item Adds Dynamic Address Reconfiguration (ADDIP)
		\item Enables SCTO to add, delete, and change existing IP addrs
			attached to an assoc. during an acrive conn.
		\item Enables support for seamless handover for mobile hosts
			that are moving between IP networks
		\item Uses ASCONF and ASCONF-ACK
		\begin{itemize}
			\item Add new IP addrs to assoc
			\item Change primary IP of assoc
			\item Delete old IP addr from assoc
		\end{itemize}
	\end{itemize}
\end{itemize}
\section{DCCP}
Motivation
\begin{itemize}
	\item UPD and TCP limitations with realtime transport of data
	\item Need to support real-time data transfers over wireless links
\end{itemize}
Overview
\begin{itemize}
	\item DCCP is novel non-reliable transport layer protocol
	\item
	\item
\end{itemize}
Features
\begin{itemize}
	\item
\end{itemize}
Datagram Format 1:
\begin{itemize}
	\item Headers - Long Sequence No.
\end{itemize}
Datagram Format 2:
\begin{itemize}
	\item Headers - Short seq no.
	\item Acknowledgements - Short seq no.
	\item Options and Data
\end{itemize}
Datagram Fields
\begin{itemize}
	\item
\end{itemize}
Packet Types
\begin{itemize}
	\item
\end{itemize}
Connection Setup
\begin{itemize}
	\item 3-way handshake
\end{itemize}
Data exchange
\begin{itemize}
	\item Two endpoints exchange Data pkts and ack pkts acking data
	\item Optionally DataAck pkts containing data and acks can be exchanged
	\item If one endpoint has no data to send it will send ack pkts
		exclusively
\end{itemize}
Connection Close
\begin{itemize}
	\item 3-way handshake
\end{itemize}
\section{Congestion Control-Related Schemes}
Drop Tail
\begin{itemize}
	\item Involves default queue mechanisms
	\item Drops all pkts exceeding queue length
	\begin{itemize}
		\item Any TCP-based receiver reports loss in ACK pkt
		\item Most often sender adapts to loss by multiplicatively
			decreasing
	\end{itemize}
	\item One loss event is very likely to be dollowed by series of loss
		events
	\begin{itemize}
		\item Little or no space in queue
	\end{itemize}
	\item If adaptive senders need some time to respond
\end{itemize}
Random Early Detection (RED)
\begin{itemize}
	\item Uses active queue management
	\item Drops pkt in intermediate node based on av. queue legnth exceeding
		a thresh
	\begin{itemize}
		\item Any TCP receiver reports loss in ACK pkt
		\item Most often sender adapts to loss by multiplicatively
			decreating rate
	\end{itemize}
	\item RED experiences mostly singular loss events
	\item Gives time to adaptive senders to respond
\end{itemize}
Early Congestion Notifications
\begin{itemize}
	\item End-End congestion avoidance mechanism
	\begin{itemize}
		\item Implementedin routers and supported by end-systems
		\item Not multimedia-specific, very TCP-specific
	\end{itemize}
	\item Uses two IP header bits
	\begin{itemize}
		\item ECT - ECN Capable Transport, set by sender
		\item CE - COngestion Experienced, may be set by routers
	\end{itemize}
	\item If pkt has ECT bit 0,
	\begin{itemize}
		\item ECN acts as RED
	\end{itemize}
	\item If pkt has ECT bit 1:
	\begin{itemize}
		\item ECN node sets CE bit
		\item Any TCP receiver sets ECN bit in ACK
		\item As result most often sender applies multi decrease
	\end{itemize}
	\item EXN-pkts never lost on un-congested links
	\item Distinction between loss and marked pkts
	\begin{itemize}
		\item TX window can decrease
		\item No pkt loss and no reTX
	\end{itemize}
\end{itemize}
Early Congestion Notification (ECN) Nonce
\begin{itemize}
	\item Optional addition to ECN
	\item Improvies robustness of congestion control
	\item Prevents receivers from ecploiting ECN to gain unfair share of
		net. BW
	\item Protects against accisdental/malicious concealment of marked pkts
		from sender
\end{itemize}
eXplicit Congestion Notiication (XCN)
\begin{itemize}
	\item Protocol for Connections with high BW-delay product
	\item Routers return explicit feedback to host
	\item Hosts use feedback from routers to change their congestion window
\end{itemize}
\section{TCP over Wireless}
Motivation
\begin{itemize}
	\item LArge percentage of traffic is reliable:
	\begin{itemize}
		\item File Trandfer (FTP)
		\item Web Traffic (HTTP)
		\item Command Based (TELNET, SSH)
	\end{itemize}
	\item TCP very popular in wired networks
	\begin{itemize}
		\item Very good congestion control
		\item Very good congestion avoidanve
	\end{itemize}
\end{itemize}
TCP in Wireless Networks
\begin{itemize}
	\item Pkt loss in wireless networks occurs due to:
	\begin{itemize}
		\item Bit errors due to wireless channel impairments
		\item Handovers due to mobility
		\item Congestion (rarely)
		\item Pkt reordering (rarely)
	\end{itemize}
	\item TCP assumes pkt loss is due to:
	\begin{itemize}
		\item COngestion in the net.
		\item Pkt reordering (rarely)
	\end{itemize}
\end{itemize}
Problems with TCP over Wireless NEtworks
\begin{itemize}
	\item Congestion avoidance can be triggered by pkt loss
	\begin{itemize}
		\item TCPs mechanisms do not respond well to pkt loss due to bit
			errors and handoffs
		\item Efficiency of TCP-based transfers suffer
	\end{itemize}
	\item Error bursts may occur due to low signal stregth or longer period
		of noise
	\begin{itemize}
		\item More than one pkt lost in TCP
		\item More likely to be detected as TO -> TCP enters slow start
	\end{itemize}
	\item Delay is often very high and variable
	\begin{itemize}
		\item RTT can be very long and variable
		\item TCPs TO mechanisms may not work well
		\item Problem exacerbated by link-level reTX
	\end{itemize}
	\item Links may be asymmetric
	\begin{itemize}
		\item Delayed ACKs in slow dirn. limit throughput in fast dirn
	\end{itemize}
\end{itemize}
Solutions for TCP over Wireless Networks
\begin{itemize}
	\item Link-Layer approaches (A)
	\begin{itemize}
		\item Hide losses not caused by congestion from the
			transport-layer sender
		\item Makes link appear to be more reliable than it is in
			reality
		\item Solns:
		\begin{itemize}
			\item Use frame reTX
			\begin{itemize}
				\item Link-level automatic reTX request (ARQ)
			\end{itemize}
			\item USe error correction codes
				\begin{itemize}
					\item Forward Error Correciton (FEC)
				\end{itemize}
			\item Use hybrid solutions
			\begin{itemize}
				\item ARQ and FEC
			\end{itemize}
		\end{itemize}
	\end{itemize}
	\item Advantages
	\begin{itemize}
		\item Requires no change to existing sender behaviour
		\item Matches layered protocol stack model
	\end{itemize}
	\item Disadvantages
	\begin{itemize}
		\item Negative TCP effect:
		\begin{itemize}
			\item Delays due to link-level TO and reTX may trigger
				TCP fast reTX
			\item TX efficiency decreases
		\end{itemize}
		\item Soln to negative TCP effect
		\begin{itemize}
			\item Make link-level protocol TCP-aware
		\end{itemize}
	\end{itemize}
	\item Example: Snoop TCP
	\begin{itemize}
		\item Advantages
		\begin{itemize}
			\item Attempts to reTX locally, suppress duplicate ack
			\item State is soft, handoff simplified
		\end{itemize}
		\item Disadvantege
		\begin{itemize}
			\item May not completely shield TCP from effects of
				mobility and wireless loss
		\end{itemize}
	\end{itemize}
\end{itemize}
Split Connection Approaches (B)
\begin{itemize}
	\item Divide single TCO conn. into two conn.
	\item Isolate wired net. from wireless net
	\item OFten split at base station or access point
	\item soln
	\begin{itemize}
		\item Use TCP on wired net
		\item Enhanced protocol over wireless net
	\end{itemize}
	\item Advantages
		\begin{itemize}
			\item Clarity of approach
			\item Each of the protocols performs best in its setup
		\end{itemize}
	\item Disadvantages
	\begin{itemize}
		\item Extra protocol OH
		\item Violates end-end semantics of TCP
		\item Complicates handoff due to state info at access point or
			base station where protocol is ``split''
	\end{itemize}
	\item Example
	\begin{itemize}
		\item Indirect TCP
	\end{itemize}
\end{itemize}
End-to-End Approaches (C)
\begin{itemize}
	\item Make sender aware that some losses are not due to congestion
	\item Avoid congestion control when not needed
	\item Solns
	\begin{itemize}
		\item Selective ack (SACKs)
		\item Explicit loss notificaiton (ELN) distinguishes between
			congestion and other losses
	\end{itemize}
	\item Advantages
	\begin{itemize}
		\item Maintains end-end semantics of TCP
		\item Intros no extra OH at base stations for protocol
			processing or handoff
	\end{itemize}
	\item Disadvantages
	\begin{itemize}
		\item Requires modified TCP
		\item May not operate efficiently e.g. for pkt reordering versus
			pkt loss
	\end{itemize}
	\item Example
	\begin{itemize}
		\item SMART
	\end{itemize}
\end{itemize}
\section{Snoop TCP}
OVerview
\begin{itemize}
	\item Link-layer protocol that snoops passing TCP data and acks
	\item Employs Snoop agent between two endpoints
	\item Data from Fixed Host to Mobile Host
	\begin{itemize}
		\item Cache unack'd TCP data
		\item Perform local reTX
	\end{itemize}
	\item Data from Mobile Host to Fixed Host
	\begin{itemize}
		\item Detect missing pkts
		\item Perform negative ack
	\end{itemize}
\end{itemize}
Architecture
\begin{itemize}
	\item
\end{itemize}
Fixed Host to Mobile Host Operation
\begin{itemize}
	\item If new pt rec. in normal TCP seq
	\begin{itemize}
		\item Add to snoop cache
		\item Forward to Mobile Host
	\end{itemize}
	\item If out of seq pkt cached earlier arrives
	\begin{itemize}
		\item Fast reTX/TO at send due to:
		\item if last\_ACK < crt\_seq\_no
		\begin{itemize}
			\item Loss in wireless link - Forward to Mobile Host
		\end{itemize}
		\item if last\_ACK > crt\_seq\_no
		\begin{itemize}
			\item Loss of previous ACK - send ACK to Fixed Host with
				Mobile Host addr and port
		\end{itemize}
	\end{itemize}
	\item If out of seq pkt not cached earlier arrives
	\begin{itemize}
		\item if seq\_no far from last\_seq\_no
		\begin{itemize}
			\item Congestion in fixed network
			\begin{itemize}
				\item Forward to Mobile Host
				\item Mark as reTX by sender
			\end{itemize}
		\end{itemize}
		\item if seq\_no close to last\_seq\_no
		\begin{itemize}
			\item Out of order delivery
		\end{itemize}
	\end{itemize}
\end{itemize}
Mobile Host to Fixed Host Operation
\begin{itemize}
	\item If new ack rec. in normal TCP operation
	\begin{itemize}
		\item Normal Case
		\begin{itemize}
			\item Clean snoop cache
			\item Update RT estimate
			\item Forward ack to Fixed Host
		\end{itemize}
	\end{itemize}
	\item if spurious ack rec.
	\begin{itemize}
		\item Discard
	\end{itemize}
	\item If duplicate ack rec.
	\begin{itemize}
		\item If pkt not in snoop cache
		\begin{itemize}
			\item Lost in fixed net.
			\begin{itemize}
				\item Forwared to fixed host
			\end{itemize}
		\end{itemize}
		\item If pkt marked as sender reTX
		\begin{itemize}
			\item Forward to Fixed Host
		\end{itemize}
		\item If unexpected (first after a pkt loss)
		\begin{itemize}
			\item Lost pjt on wireless link
			\begin{itemize}
				\item ReTX ar higher priority
			\end{itemize}
		\end{itemize}
		\item If expected (subsequent adter one lost)
		\begin{itemize}
			\item Discard
		\end{itemize}
	\end{itemize}
\end{itemize}
Advantages
\begin{itemize}
	\item Improved performance in wireless net.
	\item No change to TCP at fixed host
	\item No violaiton of end-end TCP semantics
	\item No recompiling/re-linking of existing applications
	\item Automatic fallback to standard TCP
	\begin{itemize}
		\item No need to ensure all foreign net. provide Snoop agent
	\end{itemize}
\end{itemize}
Disadvantages
\begin{itemize}
	\item Does not fully isolate wireless link errors from the fixed net.
	\item Mobile host must be modified to handle NACKs for reverse traffic
	\item Cannot snoop encrypted datagrams
	\item Cannot be used with authentication
\end{itemize}
\section{Indirect TCP (I-TCP)}
Overview
\begin{itemize}
	\item Hides pkt loss due to wireless from sender
	\item Wireless TCP can be independently optimized
	\item Good performance in case of wide-area net.
	\item reTX occurs only on bad link
	\item Faster recovery due to relativelty shortRTT for wireless link
	\item Handoff requires state transfer
	\item Buffer space needed, extra copying at proxy
	\item End-end semantics violation needs to be augmented by apllication
		level
\end{itemize}
Architecture
\begin{itemize}
	\item
\end{itemize}
Advantages
\begin{itemize}
	\item No changes to TCP at fixed hosts
	\item Wireless link errors are corrected at the TCP proxy and do not
		propagate to the fixed net.
	\item New ``wireless'' protocol affects only limited part of internet
	\item Possible further optimizaitons over wireless link
	\item Delay variane between proxy and mobile host is small -> optimised
		TCP
	\item Opportunity for header compression
	\item Opportunity for different transport protocol
\end{itemize}
Disadvantages
\begin{itemize}
	\item Loss of TCPs end-end semantics
	\item Addition of third point of failure (proxy) apart from fixed,
		mobile hosts
	\item Handover can be significant
	\item OH at proxy for per pkt processing
	\item TCP proxy must be trusted
	\item Opportunities for snooping and DOS attacks
	\item End-end IP-level privacy and auth. must terminate at proxy
	\item Proxy failure may cause loss of TCP state
\end{itemize}
\section{Other Approaches}
\end{document}
