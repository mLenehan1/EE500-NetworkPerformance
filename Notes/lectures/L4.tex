\section{Performance at Lower Network Layers}
\subsection{Network Layer - Overview}
Goal:
\begin{itemize}
	\item Transports data from src to dest, across multi. hops
	\item Every major net. component has net. layer concerns
\end{itemize}
Design Principles:
\begin{itemize}
	\item Services provided by network layershould be indep. of net.
		topology
	\item Transport layer should be shielded from member, type and topology
		of net. components
	\item Network addresses avail. to the transport layer should use uniform
		numbering plan
\end{itemize}
Service Types:
\begin{itemize}
	\item Connectionless
	\begin{itemize}
		\item e.g. IP: Internet PRotocol
	\end{itemize}
	\item Connection-oriented
	\begin{itemize}
		\item ATML Asynch. Transfer Mode Protocol
	\end{itemize}
\end{itemize}
\subsection{Internet}
\subsubsection{Characteristics}
\begin{itemize}
	\item Initial Design
	\begin{itemize}
		\item Connection-oriented
		\item Peer-peer topology
		\item Circuit swithing-based communication
		\item e.g. Telegraph and telephone
	\end{itemize}
	\item Current Design
	\begin{itemize}
		\item Best Effor Service
		\begin{itemize}
			\item More appropriate for data transfers
			\item No RT requirements
			\item End-points can adapt to net. conditions, if they
				want/need to
		\end{itemize}
		\item Connectionless packet-switching based
		\begin{itemize}
			\item Preferred to circuit-switvhing
			\item No set-up delay
			\item No blocking (fire and forget, forward if possible,
				deliver as received)
			\begin{itemize}
				\item No guarantee that any data reaches dest.
			\end{itemize}
			\item Flex. in TX bit rates
			\begin{itemize}
				\item Circuit-switching usually has few
					pre-determined bit-rates
			\end{itemize}
			\item No stable "path"
			\begin{itemize}
				\item More reliable (route around problems
			\end{itemize}
			\item More efficient use of net. resources when traffic
				bursty
		\end{itemize}
	\end{itemize}
\end{itemize}
\subsection{IP-based Network Layer}
\begin{itemize}
	\item Based on IP
	\item Connectionless: uses datagram packet switching
	\item Open design
	\item Open implementations
	\item Open standardisations process
	\item Independent of physical medium
	\item Scalable: as evidenced by its growth
	\item Extensible: protocols have evolved over time, as problems arose
		and/or req. changed
\end{itemize}
\subsection{IP Router Architecture}
\subsubsection{Router Basic Functions}
\begin{itemize}
	\item Forwarding
	\begin{itemize}
		\item Move pkts from routers input to output port
	\end{itemize}
	\item Routing
	\begin{itemize}
		\item Determine route taken by pkts from src to dest.
	\end{itemize}
	\item Routing Activity (When datalink frame arrives)
	\begin{itemize}
		\item Line card applies datalink layer logic to ensure frame is
			valid and pkt was successfully received
		\item If pkt arrival rate > routers forwarding cap. pkt queued,
			waits for processing
		\item If queue buffers full, pkt discarded
		\item If space in queue, after waiting, validity check performed
			on IP header
		\item If dest address non-local host, routing table lookup
			performed to determine how to forward pkt.
		\item Pkts classified into predefined service classes (if
			defined)
		\item TTL field decremented, new header checksum computed, pkt
			sent to approp. output port
		\item Datalink layer login on output ports line card inserts
			datalink layer header, TX pkt inside a frame
		\item If process fails, error msg sent to pkts sender
	\end{itemize}
\end{itemize}
\subsection{Internet Protocol Version 4 (IPV4)}
\subsubsection{IPv4 Header}
\begin{itemize}
	\item Ver indicates IP V num.
	\item ID is the same for all fragments of given datagram
	\begin{itemize}
		\item Fragmentation of a datagram needed when an intermediate
			net. has a max frame size too small to carry datagram
	\end{itemize}
	\item Fragmentation offset indicates place of each fragment in datagram
	\item Time-to-Live decreased by 1 at each hop
	\item Options field carrys fields that control routing, timing,
		management, security, etc.
	\begin{itemize}
		\item Padded to a multiple of 4 bytes
	\end{itemize}
\end{itemize}
\subsubsection{IPv4 Address Format}
\subsubsection{Special IPv4 Addresses}
Loopback useful for debugging/testing network software as IP pkts with loopback
addresses are processed by the machine which generated them as if they were
incoming pkts
\subsection{Internet PRotocol Version 6 (IPV6)}
\subsubsection{Motivation}
\begin{itemize}
	\item IPv4 Limitations
	\begin{itemize}
		\item Limited address space
		\begin{itemize}
			\item Despite subnetting, Network Address
				Translation (NAT)
			\item Classless Inter-Domain Routing (CIDR)
		\end{itemize}
		\item No special tratment for real-time traffic
		\begin{itemize}
			\item Despite IPv4 Type of SErvice field which is
				ignored by routers
		\end{itemize}
		\item No wide usage of security issues
		\begin{itemize}
			\item Despite IPv4 security ffeatures which are not
				widely used
		\end{itemize}
	\end{itemize}
\end{itemize}
\subsubsection{Action}
\begin{itemize}
	\item In 90s, following an open design process IPv6 was standardised
\end{itemize}
\subsubsection{ISsues}
\begin{itemize}
	\item IPv6 and v4 not compatible
	\item IPv6 deployment takes time
	\begin{itemize}
		\item Both will coexist for a long time
	\end{itemize}
\end{itemize}
\subsubsection{Features}
\begin{itemize}
	\item Increased address space
	\begin{itemize}
		\item Uses 128 bit address
		\item $2^{128}$ addresses
		\item $5*2^{28}$ addresses for each person on earth
	\end{itemize}
	\item Simple pkt header
	\begin{itemize}
		\item Routers don't do fragmentation
		\item No header checksum to be checked
	\end{itemize}
	\item Support for more options
	\begin{itemize}
		\item e.g. routing, hop-by-hop, fragmentation, etc.
		\item extension headers can be present
	\end{itemize}
	\item Support for per-flow handling and traffic classes
	\begin{itemize}
		\item Flows defined by src address, dest addres and flow num.
	\end{itemize}
	\item Mandatory authentication and security
	\begin{itemize}
		\item Internet Protocol Security (IPsec) was specially developed
			(then deployed in IPv4)
	\end{itemize}
	\item Compatability with existing TCP/IP protocol stack
	\begin{itemize}
		\item E.g DNS, TCO, UDP, OSPF, BGP, etc.
	\end{itemize}
\end{itemize}
\subsection{Other Network Layer Protocols}
\subsubsection{Motivation}
\begin{itemize}
	\item IP is in charge with data transport
	\item IP reqs. support from other transport layer protocols
	\begin{itemize}
		\item Control functions (ICMP)
		\item Multicast signalling (IGMP)
		\item Routing table setup (TIP, OSPF, BGP, PIM)
	\end{itemize}
\end{itemize}
\subsubsection{ICMP}
\begin{itemize}
	\item Internet Control Message PRotocol facilitates:
	\begin{itemize}
		\item Error reporting
		\item Simple Queries
	\end{itemize}
	\item ICMP msgs. carried by IP datagrams
\end{itemize}
\subsection{Internet Control Message Protocol (ICMP)}
\subsubsection{ICMP Message Format}
\begin{itemize}
	\item Header (4bytes)
	\item Fields:
	\begin{itemize}
		\item Type (1 byte): msg type
		\item Code (1 byte): msg subtype
		\item Checksum (2 bytes): calculated for entire msg
	\end{itemize}
	\item Payload (min 4 bytes)
	\item Content
	\begin{itemize}
		\item Related to msg type
		\item If no data, content includes 4 bytes set to 0
	\end{itemize}
	\item Note:
	\begin{itemize}
		\item ICMP msgs have min. length of 8 byte
	\end{itemize}
\end{itemize}
\subsubsection{ICMP Message Types}
\begin{itemize}
	\item Query
	\begin{itemize}
		\item Request-reply
		\item e.g. Echo request, echo reply, timestamp request,
			timestamp reply, router solicitation, router
			advertisement
	\end{itemize}
	\item Error reporting
	\begin{itemize}
		\item Informs about error in transporting data
		\item Sent by routers to hosts or routers
		\item Processed at higher layers (e.g. application)
	\end{itemize}
\end{itemize}
\subsection{Routing Algorithms}
\begin{itemize}
	\item Definition
	\begin{itemize}
		\item Determine route taken by pkts from src to dest
	\end{itemize}
	\item Desirable Properties
	\begin{itemize}
		\item Correctness
		\item Simplicity
		\item Efficiency
		\item Robustness
		\item Stability - Routing alg. reaches equilibrium in reasonable
			time
		\item Fairness
		\item Optimality
	\end{itemize}
	\item Least-Cost Routing
	\begin{itemize}
		\item Cost - A value assigned to each link in the net.
		\item Cost of a route - Sum of values of all routes links
		\item BEst route - Route with lowest cost
	\end{itemize}
	\item Meaning of cost
	\begin{itemize}
		\item 1 for each link - best route = fewest hops
		\item Financial cost of using link - best is cheapest route
		\item delau on link - bes is min.-delay route
		\item pkt tx time on link - best is max.-bw route
		\item Some comb. of these
	\end{itemize}
	\item Pkt fprwarding
	\begin{itemize}
		\item Towards the best route
	\end{itemize}
\end{itemize}
\subsubsection{Types}
\begin{itemize}
	\item Non-adaptive or Static
	\begin{itemize}
		\item Routing decisions pre-determined, not based on
			measurements (or estimates) of current net. topology and
			traffic load
	\end{itemize}
	\item Adaptive
	\begin{itemize}
		\item routing decisions may be changed when net. topology and/or
			trafic load change
		\begin{itemize}
			\item Extreme case: select new route for each pkt
			\item May get info from neighbouring routers, or from
				net. routers
			\item Routes are changed
			\begin{itemize}
				\item Periodically
				\item When topology changes
				\item When traffic load changes significantly
			\end{itemize}
		\end{itemize}
	\end{itemize}
	\item Centralised
	\begin{itemize}
		\item Routing decisions taken in centralised manner for whole
			network
	\end{itemize}
	\item Distributed
	\begin{itemize}
		\item Routing decisions taken separately by each router,
			independent from neightbours
	\end{itemize}
\end{itemize}
\subsubsection{Decentralised Routing Algorithm}
	\begin{itemize}
		\item Distance-vecor
		\begin{itemize}
			\item Each router exchanges info about entire net. with
				neighbour routers at regular intervals
			\item Neighbour - connected by direct link
			\item Regular interval - every 30 seconds
		\end{itemize}
		\item Link-state
		\begin{itemize}
			\item Each router exchanges info about neighbourhood
				with all routers in net. when there is a change
				in the topology
			\item Neighbourhood of router - set of neighbour routers
			\item each routers neighbourhood info is flooded through
				the net.
			\item Change in topology occurs hwen:
			\begin{itemize}
				\item Neighbouring router not accessibe anymore
				\item New router has been added
			\end{itemize}
		\end{itemize}
		\item Note
		\begin{itemize}
			\item Link state algorithm converges faster and
				therefore more widely used
		\end{itemize}
	\end{itemize}
\subsection{Distance Vector Routing}
Principle
\begin{itemize}
	\item Assume several LANS represented by "clouds" and routers/gateways
		by "boxes"
	\item Num. in each cloud represents net. ID
	\item Letter in each box represents router (or gateway) names
	\item Every router sends its info to its neighbours
	\item Each neighbor router adds this info to its own and send to its
		neighbours
	\item In time all routers learn about the net. struct.
	\item Each router stores info. about the net. in its routing able
	\item Routing table includes
	\begin{itemize}
		\item Net. ID = final pkt dest
		\item Cost = Num of hops from this router to final dest.
		\item Next hop = neighbouring router to which pkt should be sent
	\end{itemize}
	\item Initially each router knows net IDs of the net. to which it is
		directly connected only
\end{itemize}
\subsubsection{Router Table Update}
\begin{itemize}
	\item Distributed Bellman-Ford Algorithm
	\begin{itemize}
		\item Add 1 to cost of each incoming route (each neighbour 1 hop
			away)
		\item If new dest learned, add its info to routing table
		\item If new info received about existing dest.
		\begin{itemize}
			\item If next hop field is same, replace existing entry
				with new info even if cost is greater
			\item If next hop field not same, replace existing entry
				with new info if cost is lower
		\end{itemize}
	\end{itemize}
\end{itemize}
\subsection{Link-State Routing}
Principle
\begin{itemize}
	\item Each router sends info about is neighbourhood to every other
		router
	\item Each router updates its info about the net. based on info received
	\item In time, all routers learn about net. struct
	\item Makes use of link costs (usually a weighted sum of various factors
	\begin{itemize}
		\item e.g. traffic level, security level, pkt delay
	\end{itemize}
	\item Link cost is from router to net. connecting it to another router
	\begin{itemize}
		\item When pkt sent in LAN, every node - including router - can
			receive it
		\item No cost assigned when going from a net. to router
	\end{itemize}
\end{itemize}
Routing tables
\begin{itemize}
	\item All routers get their info about their neighbourhood by sending
		short echo pkts to their neighbours, monitoring response
	\item All routers share info about their neighbours by sending
		link-state pkts to all routers in network (flooding)
	\item Pkts include
	\begin{itemize}
		\item Advertiser: sending reouter ID
		\item Network: Dest. network ID
		\item Cost: Link cost to neighbour
		\item Neighbour: Neighbour router ID
	\end{itemize}
	\item Every router prepares a link-state pkt and floods it through the
		net.
	\item When router receives all these pkts. it can save the data in a
		link-state DB
	\item Assuming that every router receives same pkts, same content will
		be found in all link-state DBs
	\item Using info from DB, each router can fill it's routing table
\end{itemize}
\subsubsection{Dijkstra's Shortest Path Algorithm}
\begin{itemize}
	\item ID all link costs in net. either from link-state DB, or using fact
		that cost of any link from a network to a router is 0
	\item Build shortest-path spanning tree for router running the alg.
	\begin{itemize}
		\item Tree has route from router to all possible dest. and no
			loops
	\end{itemize}
	\item Router is root of its shortest-path spanning tree
	\item Node either a net. or a router: nodes connected by arcs
	\item Algorithm keeps track of 2 sets of nodes and arcs, Temp. and Perm.
	\item Initially router is in Perm. set and Temp. set contains all
		neighbour nodes of router itseld, arcs connecting them to
		router
	\item Identify Temp. node whose arc has lowest cumulative cost from
		root: move to Perm set
	\item All nodes connected to new PErm node and not already in Temp. set
		along with their arcs, moved to Temp.
	\item Any node already in Temo set has lower cumulative cost reom root
		by using a route passing through the new Perm node, this new
		route replaces the existing one
	\item Repeat until all nodes and arcs are in Perm set
	\item NoteL even if all routers link-state DBs are identical, tree
		determined by routers are different
\end{itemize}
\subsubsection{Building the Routing Table}
\begin{itemize}
	\item Once a router has found its shortest-path spanning tree it can
		build its routing table
	\item In large net. memory required to store the link-state DB and the
		computation time to calc. the link-state routing table can be
		significant
	\item In practice, since link-state pkt receptions not synchronised,
		routers may be using different link-state DBs to build their
		routing tables
	\item Result accuracy depends on how different the various routers DB
		content is
\end{itemize}
\subsection{Routing in the Internet}
\subsubsection{Centralised Routing}
\begin{itemize}
	\item Initially Internet buiilt around Core system which enabled
		interconnectivity via core gateways
	\item Routing info collected, exchanged between core Gateways using
		Gateway-Gateway protocol
	\item Routing data processed by Core and result were distrib. back to
		external routers
	\item Major weakness of model - Scalability and vulnerability
\end{itemize}
\subsubsection{Decentralised routing}
\begin{itemize}
	\item Internet build as a set of hierarchical inter-connected
		independent network groups known as Autonomous systems (AS)
	\item Two level routing
	\begin{itemize}
		\item Intra-AS: each AS responsibe for own routing
		\begin{itemize}
			\item Major protocols used in practice
			\item Routing Info Protocol (RIP) - based on distance
				vector alg.
			\item Open Shortest Path First (OSPF) - based on
				link-states alg.
		\end{itemize}
		\item Inter-AS: enables routing between AS
		\begin{itemize}
			\item Major protocol used in practice
			\item Border Gateway Protocol
		\end{itemize}
	\end{itemize}
\end{itemize}
\subsection{Multicast Routing}
\begin{itemize}
	\item Definition
	\begin{itemize}
		\item Delivery of a copy of a packet to a group of receivers
	\end{itemize}
	\item Types
	\begin{itemize}
		\item Multicast unicast
		\begin{itemize}
			\item Mult. pkts travel from src to each dest.
		\end{itemize}
		\item Multicast
		\begin{itemize}
			\item Single pktstravel on common routes
		\end{itemize}
	\end{itemize}
	\item Multicast in practice
	\begin{itemize}
		\item Requires multicast addresses for multicast groups
		\begin{itemize}
			\item Start with "10" in binary
		\end{itemize}
		\item Requires multicast enabled routers
		\begin{itemize}
			\item Maintain and pass lsit of addresses assoc. with
				multicast group address
		\end{itemize}
		\item Requires multicast routing protocols
		\begin{itemize}
			\item Distance vector mulicast routing protocols
			\item Multicast open shortest path first protocol
		\end{itemize}
		\item Requires multicast routers to know about their own
			multicast groups
		\begin{itemize}
			\item Internet Group Management Protocol
		\end{itemize}
	\end{itemize}
\end{itemize}
\subsection{Performance Issues}
\subsubsection{Distance-Vector Routings count-to-infinity problems}
\begin{itemize}
	\item Slow convergence in some conditions
	\item Slow reaction to link/router failure as info travels in small
		steps
	\item Many ad-hoc solns. have been tried, but either also fail to solve
		count-to-infinity problem or are hard to implement
\end{itemize}
\subsubsection{Link-State Routing Performance}
\begin{itemize}
	\item Link costs can be configured in OSPF (hop, reliability, delay,
		cost, bw)
	\item Large mam. req.
	\item Dijkstra alg. computations are highly processor intensive
	\item High BW req. if network topology changes often
\end{itemize}
\subsubsection{Need for Intra- Inter-AS routing}
\begin{itemize}
	\item Policy
	\begin{itemize}
		\item Inter-AS: Concerned with policies
		\item Intra-AS: Under same admin, control so policy is less
			important
	\end{itemize}
	\item Scalability
		\begin{itemize}
			\item Inter-AS: Scale for routing among large num. of
				net.
			\item Intra-AS: Scalability less of a concern
		\end{itemize}
	\item Performance
	\begin{itemize}
		\item Inter-AS: difficult to focus on performance metrics
		\item Intra-AS: highly focused on performance metrics and costs
	\end{itemize}
\end{itemize}
\subsection{Wireless Routing PRotocols}
\subsubsection{Classification}
\begin{itemize}
	\item Topology-Based Routing
	\begin{itemize}
		\item Table-Driven (proactive)
		\item On-Deman (Reactive)
		\item Hierarchical Routing
	\end{itemize}
	\item Location-based routing
	\begin{itemize}
		\item Greedy Routing
	\end{itemize}
\end{itemize}
\subsubsection{Table-Driven (Proactive) Routing}
\begin{itemize}
	\item Based on distance-vector and link-state protocols
	\item Nodes maintain routes to other nodes
	\item Periodic or event triggered route updates
	\item Relatively low latency, routes known in advanfe
	\item Higher overhead and longer route convergence
\end{itemize}
\subsubsection{On-Demand (Reactive) Routing}
\begin{itemize}
	\item Src node inits routing discovery on demand
	\item Only active routes maintained
	\item Relative reduced routing overhead
	\item Long delays when new routes fount
\end{itemize}
\subsubsection{Hierarchical Routing}
\begin{itemize}
	\item Net. divided into clusters
	\item Nodes talk to cluster head only
	\item Better scalability (descreases routing overhead), unfair use of
		resoutces
\end{itemize}
\subsubsection{Location-Based (Geographic Routing}
\begin{itemize}
	\item Routing performed according to position of node
	\item Routing overhead can be small but optimal routing may not be found
\end{itemize}
\subsubsection{Destination Sequenced Distance Vector (DSDV)}
\begin{itemize}
	\item Proactive routing protocol
	\item Each node maintains routing table with entries for each node in
		net.
	\begin{itemize}
		\item dest addr, seq num., next-hop, hop-count)
	\end{itemize}
	\item Nodes transmit pkts according to routing table
	\item Each node has seq num, updated when routing info changes (new node
		joins, line break)
	\begin{itemize}
		\item Used to avoid routing loops
	\end{itemize}
	\item Each node periodically broadcasts routing table updates
\end{itemize}
\subsubsection{Dynamic Source Routing (DSR)}
\begin{itemize}
	\item Reactive protocol
	\item Srv wamts tp TX, does not know route to dest, inits route
		discovery
	\item Route request pkt broadcast, once dest receives, send back route
		reply, in pkt header ID's each forwarding hop in next node field
	\item Entire route stored in pkt headers
	\item At nodfes, route cache used to store most recent routes
\end{itemize}
\subsubsection{Ad Hoc On-Demand Distance Vector (AODV)}
\begin{itemize}
	\item Essentially combo of DSR and DSDV
	\item DSRs on-demand mechanisms for route discovery and route
		maintenance
	\item Uses DSDVs table of precursor, next hop for each route during
		hop-by-hop routing and sequence numbers (to prevent loops)
	\item Improve DSR by keeping routing tables at nodes (pkts do not
		contain entire route)
	\item Routing table entries have lifetime in contreast to DSR cache
\end{itemize}
\subsubsection{Temporally-Ordered Routing Algorithm (TORA)}
\begin{itemize}
	\item Adaptive routing protovol for muli-hop net.
	\item Designed to min comms overhead via localization of algorighmic
		reaction to topological changes (distribd execution)
	\item Uses directed acyclic graphs instead of shortest path soln
	\item Each node assigned unique height, pkts flow from high noes to low
		nodes along path towards dest.
\end{itemize}
\subsection{Performance Issues}
\subsubsection{Throughput}
\begin{itemize}
	\item Throughput of DSDV decreases drastically with increases in
		mobnility. DSR outperforms all other protocols
\end{itemize}
\subsubsection{Overhead}
\begin{itemize}
	\item In general routing overhead increases with mobility (topology
		changes)
	\item For DSR overhead dependent on num of diff. routes
	\item For DSDV overhead higher as routing tables need to be maintained
\end{itemize}
\subsubsection{Delay}
\begin{itemize}
	\item DSDV delay lowest, constant (see routing tables)
	\item DSR high delay (see reactive protocols)
	\item TORA highest delay (see short-lived and long lived loops)
\end{itemize}
\subsubsection{Optimality}
\begin{itemize}
	\item DSDV and DSR find optimal paths
	\item TORA and AODV use suboptimal paths even under low mobility
\end{itemize}
\subsection{Quality of Service Support}
\subsubsection{Buffering}
\begin{itemize}
	\item Significant traffic burstiness when Tx over net.
	\item Buffering reduces loss, enables control over Tx rate
\end{itemize}
\subsubsection{Packet Scheduling}
\begin{itemize}
	\item Enables selection of packets for differentiated Tx
	\item Arrival based: First in First Out (FIFO), Last In First Out (LIFO)
	\item Priority based: Src or pkts have different priority
	\item Weight based: Weighted Fair Queueing (WFQ)
\end{itemize}
\subsubsection{Traffic Shaping}
\begin{itemize}
	\item Controls flow of data
	\item Time-based: Leaky bucket alg
	\item Token-based: Token bucket alg
\end{itemize}
\subsubsection{Admission Control}
\begin{itemize}
	\item Enables access if certain performance metrics are met
	\item Otherwise refuses admission
	\item Maintains certain level of performance e.g. quiality
\end{itemize}
\subsection{Traffic Engineering}
\subsubsection{Motivation}
\begin{itemize}
	\item Network traffic highly dynamic
	\item Network resources variable
	\item No network control
	\item No guaranteed QoS
	\item No efficient use of network resources
	\item No guaranteed security, reliability, resilience, etc.
\end{itemize}
\subsubsection{Definition}
\begin{itemize}
	\item TE is concerned with optimizing performance of telecomms network
		by dynamically analyzing, predicting, and regulating the bhavour
		of dat TX over that network
\end{itemize}
\subsubsection{Goal}
\begin{itemize}
	\item Optimisation in terms of efficiency (i,e, costs and quality)
\end{itemize}
\subsubsection{Major Solutions}
\begin{itemize}
	\item Intergrated Services
	\begin{itemize}
		\item Focus on providing QoS delivery guarantees per-flow (using
			resource reservation: RSVP)
		\item Concerns on: complexity, scalabity, business model, etc.
	\end{itemize}
	\item Differentiated Services
	\begin{itemize}
		\item Focus on providing QoS support per-class
		\item Routers on the network differentiate traffic treatment
			based on it's class, ensuring preferential treatment for
			higher priority traffic
		\item No advance setup, no reservation, no negotiations for each
			flow, easier to implement
	\end{itemize}
	\item Other solutions: MLPS
\end{itemize}
\subsection{Performance of Datalink and Physical Network Layers}
\begin{itemize}
	\item Wireless PANs (BT - IEEE802.15)
	\begin{itemize}
		\item v. low range
		\item wireless connection to printers etc
	\end{itemize}
	\item Wireless LANs (WiFi - IEEE 802.11)
	\begin{itemize}
		\item Infrastructure as well as ad-hoc net. possible
		\item Home/office net.
	\end{itemize}
	\item Wireless MANs (WiMAX - IEEE 802.16)
	\begin{itemize}
		\item Large scale network
		\item Base station-based infrastructure
	\end{itemize}
\end{itemize}
\subsection{WiMax IEEE 802.16}
\begin{itemize}
	\item Group formed in 98
	\item Standards
	\begin{itemize}
		\item Air-interface for wireless broadband
		\item Line of Sight comms
		\item Ooperates in 10-66GHz range
	\end{itemize}
	\item 802,16a amendment
	\begin{itemize}
		\item Included Non LOS version in 2-11 GHz freq. band
		\item PHY layer uses orthogonal freq. division multiplexing
		\item MAC layer supports Orthogonal Frequency Division Multiple
			Access
	\end{itemize}
	\item 802.16d
	\begin{itemize}
		\item Further amended standard
		\item Formed basis for first WiMax soluions
	\end{itemize}
        \item Above standards supported fixed wireless applications
	\begin{itemize}
		\item No mobility support
	\end{itemize}
	\item 802.16e - 2005
	\begin{itemize}
		\item added support for mobility and improved performance
		\item Enable soft and hard handover between base stations
		\item Introduce scalable OFDMA
		\begin{itemize}
			\item Enables higher sppectrum efficiency in wide
				channels
			\item Cosst reduction in narrow channels
		\end{itemize}
		\item Improces coverage using
		\begin{itemize}
			\item Antenna diversity schemes
			\item Hybrid ARQ (hARQ)
		\end{itemize}
		\item Improving capacity and coverage using
		\begin{itemize}
			\item Adaptive Antenna Systes (AAS)
			\item Multiple Input Multiple Output(MIMO) tech
		\end{itemize}
		\item Into'd high performance coding techniques to enhance
			security and NLOS performance
		\begin{itemize}
			\item Turbo coding
			\item Low density parity check
		\end{itemize}
		\item Intro's downlink sub-channelization alling admins trade
			coverage for capacity or vice versa
		\item Increases resistance to multipath interference using
			enhances FFT alg. which can tolerate larger delay
			spreads
		\item Adds extra QoS classs (enhanced rtPS) more appropriate for
			VoIP apps
	\end{itemize}
\end{itemize}
\subsection{WiMax 802.16e Service Classes}
\subsubsection{Unsolicited Grant Service (UGS)}
\begin{itemize}
	\item Fixed size pkt carried periodically without requiring explicit
		req. for bw allocation every time. Real-time high bw (T1) CBR
		applications (e.g. VoIP)
\end{itemize}
\subsubsection{Extended Real-Time Poling Service (ertPS)}
\begin{itemize}
	\item Newly intro'd scheduling service in 802.16e complement periodic bw
		allocations with possibilities for mobile stations to req.
		additional resources during original allocation. Supports
		applications whose bw req. vary in time (e.g. VoIP, streaming)
\end{itemize}
\subsubsection{Real Time Polling Service (rtPS)}
\begin{itemize}
	\item Intro'd to support real time services with variable zide data pkts
		fenerated periodically, such as MPEG video delivery
		applications. Frequent unicast polling opps are provided such as
		movile stations can req. bw and satisfy their timing req.
\end{itemize}
\subsubsection{Non-Real-Time Polling Service (nrtPS)}
\begin{itemize}
	\item Similar with rtPS, unicast polling opps less freqent. Contention
		based polling can also be used to req. bw resources
\end{itemize}
\subsubsection{Best Effort Service (BE)}
\begin{itemize}
	\item Designed for services with no strict QoS requirements such as
		email and web apps. Mobile stations use contention based polling
		to request resources
\end{itemize}
\subsection{WiFi IEEE 802.11}
\begin{itemize}
	\item First std published in 97 for WLAN comms
	\item Since, various extensions proposed to address different issues -
		higher bit rate, QoS support, security
	\item Tech gained popularity because of low deployment and maintenance
		cost, as well as relatively high bitrate
	\item IEEE 802.11 - 1997
	\begin{itemize}
		\item supports data rates up to 2Mbps, initially developed for
			best effort traffic only
		\item Each host connects to an IEEE802.11 access point
		\item Wireless medium shared with other nodes associated with
			same AP point
		\item Contention for medium access which determines increased
			collision rates and consequently lower data rates
			especially when num of mobile hosts involved in
			sumultaneous data comms increases
		\item IEEE 802.11 MAC layer provides mech. for medium access
			coordination:
		\begin{itemize}
			\item Distributed Coordination Function (DCF) -
				distributed
			\item Point coordination Function (PCF) - partly
				centralised
		\end{itemize}
	\end{itemize}
	\item IEEE 802.11b
	\begin{itemize}
		\item Increased max data rate to 11Mbps, operating in 2.4 GHz
			freq. band
	\end{itemize}
	\item IEEE 802.11g
	\begin{itemize}
		\item Increase max. data rates to 54Mbps
	\end{itemize}
	\item IEEE 802.11a
	\begin{itemize}
		\item Data rates up to 54Mbps operating in 5GHz freq. band
	\end{itemize}
	\item IEEE 802.11e
	\begin{itemize}
		\item QoS extesion provided by two new mechanisms
		\begin{itemize}
			\item Hybrid Coordination Function (HCF) - PCF extension
			\item Enhanced Distributed Coordination Function (EDCF)
				- DCF extension
		\end{itemize}
	\end{itemize}
\end{itemize}
\subsubsection{Other 802.11 extensions}
\begin{itemize}
	\item IEEE 802.11n
	\begin{itemize}
		\item Higher bitrates up to 600Mbps
		\item QoS support similar with 802.11e
	\end{itemize}
	\item IEEE 802.11p
	\begin{itemize}
		\item Wireless comms in vehicular environments
		\item Short to medium range comms at high data transfer rates
	\end{itemize}
	\item IEEE 802.11ac VHT
	\begin{itemize}
		\item Offers data rates up to 1Gbps for low velocity mobile
			hosts
	\end{itemize}
\end{itemize}
\subsection{WiFi IEEE 802.11 Issues}
\subsubsection{Hidden Station Problem}
\begin{itemize}
	\item Consider that station B has TX range indicated by left oval, C by
		right oval. Any station in these ranges can hear TX from B and C
		respectively
	\item Station C, outside TX range of B cannot hear B, likewise B cannot
		hear C. Station A in range of both. Assuming B TX to A, C cannot
		hear B, believes medium is free, also TX data to A. Neither TX
		successful
	\item RST and CTS frames introduced to solve this problem. Before
		sending, B sends RTS to A (includes duration of TX). A hears RTS
		and replies with CTS which also includes TX duration info. As C
		is in range of A it gets CTS msg, learns of hidden station will
		be using channel, refrains from TX.
\end{itemize}
\subsection{WiFi IEEE 802.11e}
\subsubsection{QoS Support}
\begin{itemize}
	\item Access Class (AC)
	\begin{itemize}
		\item AC\_VO (Voice), AC\_VI (video), AC\_BE (best effort) and
			AC\_BK (Background)
	\end{itemize}
	\item Transmission Opportunity (TxOP)
	\begin{itemize}
		\item Time duration dring which station is allowed Tx turst of
			data frames
	\end{itemize}
	\item Arbitration Interframe Space (AIFS[AC])
	\begin{itemize}
		\item Period of time a wireless node has to wait before alloewd
			TX next frame
		\item Dependent on access class
	\end{itemize}
	\item Contention Window (CWmin, CWmax)
	\begin{itemize}
		\item Deoendent on access class
	\end{itemize}
\end{itemize}
\subsection{WPAN}
\subsubsection{IEEE 802.15}
\begin{itemize}
	\item Bluetooth
	\begin{itemize}
		\item Interconnects various protale devices and their
			accessories
		\item 2.4GHz band
		\item Data rates of up to 1Mbps (BT v1.0) and up to 3Mbps (v2.0)
		\item Future rates expected to be between 53Mbps and 480Mbos
		\item IEEE802.15.1 based on BT v1.1
	\end{itemize}
	\item IEEE 802.15.4
	\begin{itemize}
		\item Low-range, low-power wireless network comms
		\item Based on this standard, Zigbee protocol defines the
			network layer specialized on ad-hoc networking and the
			application layer targeting wireless sensor networks as
			well as other monitoring and control applications
		\item IEEE 802.15.4/Zigbee offers data rates up to 250 Kbps in
			the 2.4GHz band
	\end{itemize}
	\item Ultra-Wideband (UWB)
	\begin{itemize}
		\item WiMedia Alliance defined UWB wireless comms tech
			sipporting wide range of data rates from 53Mbps to
			480Mbps over short range using low power transceivers
		\item PHY/MAC protocols developed by WiMedia became ECMA 368
			standard and later on ISO/OEC 26907
	\end{itemize}
	\item Wibree
	\begin{itemize}
		\item Ultra-low power wireless net. comms tech
		\item Ranges up to 10m and data rates of 1Mbps
	\end{itemize}
\end{itemize}
