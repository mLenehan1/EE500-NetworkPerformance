\section{Network Applications}

\subsection{Multi-programming}

Uniprogramming:
\begin{itemize}
	\item Originally only 1 usr program on computer at a time
\end{itemize}

Multi-Programming
\begin{itemize}
	\item More user programs run on same computer at same time
\end{itemize}

Process:
\begin{itemize}
	\item Running program along with all resources that its code can affect
\end{itemize}

PRocess Context:
\begin{itemize}
	\item Represent the process state
	\item Includes code, variables, open files, memory contents, stack, etc.
\end{itemize}

\subsection{Multi-Tasking}

Multi-programming problem
\begin{itemize}
	\item Computer has limited no. of CPUs
	\item Many user programs running sim. need CPU time
\end{itemize}

Muti-programming solution:
\begin{itemize}
	\item Multi taskingL illusion of concurrecy (parallel execution)
\end{itemize}

Multi-Tasking:
\begin{itemize}
	\item SW approach to achieve parallel processing
	\item Dast switching CPUamong different processe
	\item At any given time one process only running on one CPU
	\item All processes seem to run in paprallel
\end{itemize}

\subsection{PRocesses}

Process States:
\begin{itemize}
	\item A process may be in one of three states:
\end{itemize}

Running:
\begin{itemize}
	\item Using the CPU
\end{itemize}

Blocked
\begin{itemize}
	\item Unable to run until some external event happens (CPU could be
		free)
\end{itemize}

Ready (Runnable)
\begin{itemize}
	\item Temp stopped to let other process run on CPU
\end{itemize}

Process Transitions
\begin{enumerate}
	\item Occurs when a process cannot continue as it is waiting for
		external event
	\item Caused by process scheduler when it decides to stop temp the
		execution of current process and give another chance to run
	\item Caused by process scheduler when it decides to give a ready
		process the chance to run
	\item Occurs when the external event the blocked process was waiting
		for happens
\end{enumerate}

\subsection{Scheduling Algorithms}

\begin{itemize}
	\item Determing when to stop one process/give CPU time to another
	\item Get involved when 1+ process in READY state and sys must decide
		whihc to run first
	\item Could be voluntary ("Non-pre-emptive scheduling") or forced
		("Pre-emptive scheduling")
	\item Pre-emptive scheduling can be performed according to different
		policies
	\begin{itemize}
		\item Priority scheduling
		\item First-come First-served
		\item Shortest job first
		\item Shortest remaining job first
		\item Round-robin
	\end{itemize}
\end{itemize}

\subsection{Threads}

\begin{itemize}
	\item A sequence of a program that erforms certain task and exec within
		a process
	\item Has its own stack
	\item Shares memory and data with other threads within the same process
	\item Properties:
	\begin{itemize}
		\item Can have different priorities
		\item Can run in preemptce mode (OS interrupts thread execution
			at regular intervals to give execution time to other
			threads)
		\item Cooperative mode (thread can control CPU for as long as it
			needs)
	\end{itemize}
\end{itemize}

\subsection{Multi-Threading}

\begin{itemize}
	\item Provides another level of parallelism for task execution, with
		less overhead
	\item Enables different tasks to be performed in parallel using common
		data and resources
	\item Thread context switching is less complex and faster than process
		context swithcing
	\item Process consists of many threads each running at same time within
		process context, performing unique tasks
	\item Suggestions
	\begin{itemize}
		\item Thread which manages time-critical tasks should be given
			high priority than other threads
		\item GUI and various data processing should be allocated
			different threads
		\item No of threads should be kept to minimum, not overload sys
	\end{itemize}
\end{itemize}

\subsection{Inter-Thread Communications}

\begin{itemize}
	\item Focuses on exchanging data between different threads
	\item As mostly performed using common data variables (share processes
		data space)
	\item As more than one thread executing in parallel may access and
		modify the data, the results may be unpredictible
	\item Solution:
	\begin{itemize}
		\item Only 1 thread allowed to modify data at a time (inter
			thread sync)
		\item Mechanisms to enable inter-thread comms include:
		\begin{itemize}
			\item Shared memory
			\item Semaphores
			\item Message passing
			\item Signals
			\item Named pipes
		\end{itemize}
	\end{itemize}
\end{itemize}
