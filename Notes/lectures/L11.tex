\section{Next Genneration Networking}

\subsection{Mobile Key Features}

\begin{itemize}
	\item High performance: Processing and storage
	\item High quality multimedia
	\begin{itemize}
		\item 4K UHD video player/recorder
		\item High resolution cameras
	\end{itemize}
	\item Long battery Life
\end{itemize}

\subsection{Fundamental Goal: COnnectivity}

\subsection{1G}

\begin{itemize}
	\item Introd in 1980s
	\item Established foundation of mobile net
	\item Used analog radio signals
	\item Used Freq Div Multi Acces (FDMA)
	\begin{itemize}
		\item 1 user per channel
		\item Neighbouring cells assigned diff freq to avoid
			interference
	\end{itemize}
	\item AMPS, NMT, TACS
\end{itemize}

\subsection{1G Limitations}

\begin{itemize}
	\item Large freq gap req. between users to avoid interference
	\begin{itemize}
		\item Inefficient use of spectrum
		\item Scalability Issues
	\end{itemize}
	\item Analog Phones large/heavy, power inefficient, expensive
\end{itemize}

\subsection{2G}

\begin{itemize}
	\item Introd in 1990s
	\item Used digital readio signals
	\item Combines FDMA with Time Div Multi Access (TDMA)
	\begin{itemize}
		\item Multi Users per channel
		\item Small, power saving, inexpensive phones
	\end{itemize}
	\item Introduces data service for mobile
	\begin{itemize}
		\item SMS
		\item MMS
	\end{itemize}
	\item D-AMPS, GSM
\end{itemize}

\subsection{TDMA Limitations}

\begin{itemize}
	\item Still req. large freq gaps to reduce interference
	\item Rigid slots assignment, whether or not users have voice/data to
		send
	\item Potential for call drop when switching channels between adj cells
\end{itemize}

\subsection{Code Division Multiple Access (CDMA}

\begin{itemize}
	\item Each subscriber has unique code
	\item Multiple simultaneous users per channel
	\item Same radio channel can be used in ad cells
	\item Spectrum allocated to inactive users used to support new users
	\item Established foundation of 3G
\end{itemize}

\subsection{Why 3G?}

\begin{itemize}
	\item With 2G more perope had subscriptions
	\item Advances in device tech lead to era of smartphones
	\item Internet widely adopted at homes/offices
	\item People wanted more than voice and simple data
\end{itemize}

\subsection{3G}

\begin{itemize}
	\item Introd in early 21st century
	\item Accommodate web-based apps and phone-based audio and video files
	\item 2 competing standards
	\begin{itemize}
		\item CDMA2000/EV-DO
		\item WCDMA/HSPA
	\end{itemize}
\end{itemize}

\subsection{CDMA2000/ED-VO vs. WCDMA/HSPA}

\subsection{EV-DO and HSPA Benefits}

\begin{itemize}
	\item Delivered achievable throughput > 2Mbps
	\item Reduced operator cost for device services
	\item Continuous evolution for enhanced services
\end{itemize}

\subsection{4G: Faster and better broadband experience}

\begin{itemize}
	\item Introd in early 2010s
	\item Compleents 3G to boost data cap
 	\begin{itemize}
		\item $\ge$1Gbps for stationary users
		\item $\ge$100Mbps for mobile users
 	\end{itemize}
	\item LTE to bridge gap
	\begin{itemize}
		\item Labeled 4G because provides substantial improvement over
			3G
		\item Net started advertising connections as 4G LTE
	\end{itemize}
	\item Multimode 3G/LTE is foundation for successful 4G LTE
\end{itemize}

\subsection{4G LTE}

\begin{itemize}
	\item Allows for downloading, browsing, streaming, gaming faster than
		ever
	\item 2 goals
	\begin{itemize}
		\item Connect daster
		\begin{itemize}
			\item Wider channels (up to 20MHz) with OFDMA
			\item More antennas (2x2 MIMO mainstream)
		\end{itemize}
		\item Connect real-time
		\begin{itemize}
			\item Simpolified core network (flat IP architecture)
			\item Low latency
		\end{itemize}
	\end{itemize}
\end{itemize}

\subsection{4G LTE TDD}

\begin{itemize}
	\item Time Division Duplexing
	\item Single freq for up/downloading
	\begin{itemize}
		\item Up/download ratio can be changed dynamically depending on
			data needs
		\item Works better with high freq (from 1850MHz to 3800MHz)
		\begin{itemize}
			\item Cheaper to acces, less traffic
			\item Overlap with WiMax bands
			\begin{itemize}
				\item Easy upgrade of WiMax to LTE-TDD
			\end{itemize}
		\end{itemize}
		\item Popular for ISP with no 2G/3G services
	\end{itemize}
\end{itemize}

\subsection{4G LTE FDD}

\begin{itemize}
	\item Frequency Division Duplexing
	\item Uses paired frequencies to up/download data simultaneously
	\begin{itemize}
		\item Works better at lower frequencies (from 450MHz to 3600MHz)
		\item Efficient in syummetric traffic
		\item Easier and efficient radio planning
		\item Popualar for cell operators having established 2G/3G
			services
		\end{itemize}
\end{itemize}

\subsection{LTE Advanced}

\begin{itemize}
	\item Considered ``real'' 4G tech
	\item Data transfer speeds $\ge$ 3xLTE (300Mbps)
	\begin{itemize}
		\item Carrier/channel aggreagation
		\begin{itemize}
			\item 20 MHz channels cannot provide 1Gbps
				throughput
			\item Increase bandwidth used by aggreadgating
				carriers
			\item Up to 5 20MHz carriers can be aggregated,
				up to 3 in practive
			\item When carriers aggreagated, can either
				primary or secondary
		\end{itemize}
	\end{itemize}
\end{itemize}

\subsection{3G and 4G Evolution}

\subsection{5G}

\begin{itemize}
	\item Next major phase of mobile telecoms and wireless systems
	\item Will provide higher speeds, greater cap, lower latency
	\item Will be capable of supporting billions of connected devices
	\item Will be a heterogeneous network of many wireless technologies
\end{itemize}

\subsection{5G: General Requirements}

\subsection{5G: PErformance Requirements}

\subsection{5G: Use Cases}

\subsection{Enhancing Mobile Broadband Experience}

\subsection{Connecting Massive Number of Devices}

\subsection{Enabling Critical Control of Remote Devices}

\subsection{5G: Spectrum}

\begin{itemize}
	\item Below 1GHz
	\begin{itemize}
		\item Very good coverage, cannot enable dastest data rates
	\end{itemize}
	\item Between 1 and 6GHz
	\begin{itemize}
		\item Mix of coverage and cap.
		\item Carrier aggregation needed to support 5G data rates
	\end{itemize}
	\item Beyond 6GHz
	\begin{itemize}
		\item Extremely fast data rate, very limited coverage
	\end{itemize}
\end{itemize}

\subsection{5G Standardization (ITU and 3GPP)}

\subsection{Internet of Things (IoT)}

\begin{itemize}
	\item ``A global infrastructure for the information society, enabling
		advanced services, by interconnecting (physical and
		virtual)thigns based on existing and evolving interoperable
		information and communication technologies'' - ITU-T, 2012
	\item Thing are objects capable of being ID'd and integrated into comm
		lauer
	\begin{itemize}
		\item Physical things: devices with comms capability eg.
			surrounding environment, sensors, actuators, electrical
			equipment
		\item Virtual things that are capable of being stired, processed
			and accessed, e.g. multimedia content, Facebook and
			twitter accounts, etc
	\end{itemize}
\end{itemize}

\subsection{IoT Characteristics}

\begin{itemize}
	\item Interconnectivity
	\begin{itemize}
		\item Devices should be able to communicate with other devices
			to offer remote control, monitoring, sensing
		\item Control temp at home while at office, car updateing maps
			for nav sys while in garage, etc
	\end{itemize}
	\item Heterogeneity
	\begin{itemize}
		\item Devices with diff HW and connected through diff tech
		\item Interoperability should be supported
	\end{itemize}
	\item Self-adapting
	\begin{itemize}
		\item V. Large number of connected devices
		\item Huge amounts of data $\to$ Big Data
	\end{itemize}
\end{itemize}

\subsection{IoT Reference Model}

\begin{itemize}
	\item Devices TX sensed data to cloud via diff networks (LTE, WiFi,
		Zigbee)
	\item Data processed and stored at cloud computing infrastructure
	\item Cloud deploys processing techniques to extract ingo and provide
		services to users
	\item Security modules
	\begin{itemize}
		\item Decides which services to provide to which users
		\item How data from/to clout is TX and how its stored
	\end{itemize}
\end{itemize}

\subsection{Edge Computing}

\begin{itemize}
	\item Goal: Optimize cloud computing sys
	\item IdeaL distib data provessing and storage by putting resources near
		src of data
	\item Advantages:
	\begin{itemize}
		\item Reduce comms BW needed between sensors nad cloudd
		\item Reduce latency
	\end{itemize}
\end{itemize}

\subsection{IoT Applications}

\begin{itemize}
	\item Smart citiesL manage assets and resources efficiently via data
		collection from citizens and IoT devices
	\item Intelligent TransportationL enhance road safecty and traffic
		management
	\item Smart energy: designm new ways to save energy
	\item Smart agri: enhance cap of agri sys
	\item Smart Healthcare: Improce clinical care, research and puoblic
		health
\end{itemize}

\subsection{Machine-to-Machine Communicaiton (M2M)}

\begin{itemize}
	\item Direct comms between devices using any comms channels w/o manual
		assistance of humans
	\item Often used for remote monitoring: warehouse management, traffic
		control, robotics, fleet management, telemetry
	\item Key components: sensors, RFID, non-IP based net (e.g. Zigbee, BT),
		and a SW to help remote receiver interpret data and make
		decisions
	\item Not standardized as many M2M systems are built to be task or
		device specific
\end{itemize}

\subsection{IoT vs. M2M}

\begin{itemize}
	\item M2M
	\begin{itemize}
		\item almost synonymous with isolated systems of sensors
	\end{itemize}
	\item IoT
	\begin{itemize}
		\item Try to marry disparate systems into wider sys view to
			enable new applications
	\end{itemize}
	\item When considere in larger context with IP connectivity, M2M becomes
		IoT
\end{itemize}

\subsection{Software-Defined Networks (SDN)}

\begin{itemize}
	\item Trad net cannot cope up and meet net req.
	\begin{itemize}
		\item Very exensive (CAPEX and OPEX)
		\item Cannot be dynamically config.
		\begin{itemize}
			\item Manual config
		\end{itemize}
		\item Net elements lack ability to customize features
		\item Do not support dynamic scalability
	\end{itemize}
	\item Decouples control plane from forwarding plane
	\item Control plane hosted in centrlized entity, called SDN controller
	\begin{itemize}
		\item Has global view of network
	\end{itemize}
	\item Forwarding plane kept at switches
	\begin{itemize}
		\item Also called data plane
	\end{itemize}
	\item Extremely scalable, easily config's/managesd, programmable and
		inexpensive
\end{itemize}

\subsection{SDN architecture}

\begin{itemize}
	\item Application plane
	\begin{itemize}
		\item Programs that proide instructions and req.
		\item Abstract view of net
	\end{itemize}
	\item Control plane
	\begin{itemize}
		\item Relays app layer instructions and req. to net component
		\item Collects info from net devices and comms it to app layer
	\end{itemize}
	\item Data Plane
	\begin{itemize}
		\item Forwarding data and collecting net state info
	\end{itemize}
\end{itemize}

\subsection{SDN and OpenFlow}

\begin{itemize}
	\item OpenFlow: open protocol that provides a standard interface for
		programming data plane of switches
	\begin{itemize}
		\item PRovides definition of abstract swithes so that swithces
			of diff vvendors can be manages by single protocol
	\end{itemize}
	\item Considered as enabler to SDN
	\item Based on Ethernet switches that has 2 components:
	\begin{itemize}
		\item Secure channel: interface that connects each OpenFlow
			swithc to controller
		\item Flow tables: define what actions should be applied to
			packets received by OpenFlow switches
	\end{itemize}
\end{itemize}

\subsection{OpenFlow Tables}

\begin{itemize}
	\item Various fields
	\begin{itemize}
		\item Rule: matching criteria against which packets are compared
		\item Action: Instructions to be executed when packaet matches
			entry
		\item Stats: keeps track of num of pkts that match the entry
		\item PRiority: enables switch to select action with highest
			priority when multiple matches are found
		\item HArd TO: time adter which entry will be removed
	\end{itemize}
\end{itemize}

\subsection{Network Functions Virtualization (NFV)}

\begin{itemize}
	\item Current networks have custom hardware appliances for each network
		function (e.g. switches, routers, firewalls, load balancers,
		servers, etc.)
	\begin{itemize}
		\item Complex, hard to maintain
	\end{itemize}
	\item NFV decouples net functions from dedicated HQ to run them on
		standard servers and switches
	\item Advantages:
	\begin{itemize}
		\item Standard HW architecture
		\item V. flexible, scalable and inexpensive
		\item Reduced power consumption
		\item Test new apps
	\end{itemize}
\end{itemize}

\subsection{NFV Framework}

\begin{itemize}
	\item VNFs (Virtualized NEtwork Functions)
	\begin{itemize}
		\item SW used to virtually create the various net functions
			(switch, firewall, etc.)
	\end{itemize}
	\item NFVI (NFV Infrastructure)
	\begin{itemize}
		\item All HW and SW components contained within environment in
			which VNFs are deployed
		\item Can be located across several physical locations
	\end{itemize}
	\item NFV-MANO
	\begin{itemize}
		\item Functional blocks that run and manage NFVI and VNFs
	\end{itemize}
\end{itemize}

\subsection{Difference between NVF and SDN}

\begin{itemize}
	\item NVF and SDN very closely linked, not the same
	\item SDN replaces standardised netwroking protocols with centralised
		control
	\begin{itemize}
		\item Central view for more efficient implementation and running
			of the net serviecs
	\end{itemize}
	\item NFV replaces proprietary net HW with SW that can run on standards
		HW
	\begin{itemize}
		\item Optimizing the network services themselves
	\end{itemize}
	\item NFV and SDN are complementary, but not dependent on one
		another
	\begin{itemize}
		\item One can exist without the other
	\end{itemize}
\end{itemize}
