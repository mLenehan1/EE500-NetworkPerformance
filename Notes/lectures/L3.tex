\section{Introduction}
\subsection{Current Environment}
\begin{itemize}
	\item Cause:
	\begin{itemize}
		\item Advances in mobbile devices, easy to use, affordable and
			powerful
		\item People can connect to Internet anytime, anywhere
		\item Popularity of video-sharing websites
	\end{itemize}
	\item Effect:
	\begin{itemize}
		\item Mobile users demands increasing
		\item Exponential growth in video traffic
		\item Explosion in data traffic
	\end{itemize}
	\item Problems:
	\begin{itemize}
		\item Application requirements
		\item Multiple Device Types
		\item Different technologies
		\item Different User preferences (cost, energy, quality)
	\end{itemize}
	\item Solution
	\begin{itemize}
		\item Coexistence of multiple technologies
		\item Deployment of different radio access technologies in
			overlapping areas
		\item Accomodate more and more subscribers
	\end{itemize}
	\item Challenges:
	\begin{itemize}
		\item Offer always best connectivity to the interet for mobile
			users
		\item Ne on best availabel radio access network
		\item Network optimization especially for video traffic
		\item Provide continuous and smooth video streaming, minimal
			delay, jitter, and packet loss
		\item Avoid degredation in video quality and user experience
	\end{itemize}
\end{itemize}
\subsection{What is QoS?}
\begin{itemize}
	\item What is quality?
	\begin{itemize}
		\item "The totality of characteristics of an entity that bear on
			its abuility to satisfy states and implied needs" ISO
			8402
		\item "Degree to which a set of inherent characteristics fulfils
			requirements" ISO 9000
	\end{itemize}
	\item What is QoS?
	\begin{itemize}
		\item A subset of overall quality
		\item "The collective effect of service performance which
			determine the degree of satisfaction of a user of the
			service" ITU-T Rec. E.800
	\end{itemize}
\end{itemize}

\section{Service Level Agreement}
\begin{itemize}
	\item Contract between ISP and Client
	\item ISP gives guarantees for delivery of service
	\item Service Level Objectives (SLO)
	\begin{itemize}
		\item Goals needed to be met for service
		\item Used to specify QoS desired
	\end{itemize}
	\item Service Level Guarantees (SLG)
	\begin{itemize}
		\item Promise to meet SLOs
	\end{itemize}
	\item Service Level Managements (SLM)
	\begin{itemize}
		\item Approach of ISP for operation and delivery of services
		\item Integrated management of functionalities in SLA life cycle
	\end{itemize}
\end{itemize}
\subsection{Service Level Objectives}
\begin{itemize}
	\item QoS Parameters
	\begin{itemize}
		\item Instance to represent QoS to customers
		\item Different according to type of service
	\end{itemize}
	\item Generic QoS params required in network service:
	\begin{itemize}
		\item Availability, Delivery, Latency, Bandwidth, MTBF (Mean
			Time Between Failures), MTTR (Mean Time to Recover)
	\end{itemize}
\end{itemize}
\subsubsection{QoS Parameters}
\begin{itemize}
	\item Availability
	\begin{itemize}
		\item $\%$ of feasibility of service in every service request
		\item key parameter for customers
	\end{itemize}
\item Delivery
	\begin{itemize}
		\item Converse of packet loss
		\item $\%$ of each service delivered without packet loss
	\end{itemize}
\item Latency
	\begin{itemize}
		\item $\delta t$ packet to travel from service access point
			(SAP) to target and back
		\item includes transport t and queuing delay
	\end{itemize}
\item Bandwidth
	\begin{itemize}
		\item Used/Available capacity - Stated in SLA
	\end{itemize}
\item MTBF - Mean Time Between Failure
\item MTTR - Mean Time To Recover
	\begin{itemize}
		\item Avg. t device/sys takes to recover from failure
	\end{itemize}
\end{itemize}
\subsection{Service Level Guarantees}
\subsubsection{Customer requirements of QoS}
\begin{itemize}
	\item Focus on user-percieved effects
	\item Not depend on assumptions of internal net design
	\item Take into account all aspects of service from cusomers PoV
	\item Assured to user by ISP
	\item Described in net-independent terms, creates common lang.
		understandable by both user and ISP
	\begin{itemize}
		\item ITU-TG.1010
	\end{itemize}
\end{itemize}
\subsubsection{QoS Offered by the Service Provider}
\begin{itemize}
	\item QoS metrics for web browsing
	\item Requirements:
	\begin{itemize}
		\item Mainly influenced by response time (!> 5s)
		\item Delay $<400ms$ expected for best effort net traffic
		\item Jitter not applicable to HTTP
		\begin{itemize}
			\item Little impact on txt/picture web browsing
		\end{itemize}
		\item Data rate \& required b.w. $<30.5kbps$
		\item Expected loss rate \& error rate 0 since HTTP is
			reliable
		\begin{itemize}
			\item Error reTX
		\end{itemize}
	\end{itemize}
	\item QoS Metrics for Video
	\begin{itemize}
		\item Under diff. codec tech and quality req. diff. req. for net
			TX
		\item VCR quality MPEG-1 stream:
		\begin{itemize}
			\item B.W $1.2-1.5Mbps$
			\item Jitter recommended $<100ms$ for broadcast quality
			\item Residual bit error rate $<10^5$ for broadcast
				quality stream using compressed format
			\item Loss/Error rates $<10^5$
		\end{itemize}
		\item HDTV quality MPEG-2 video streams:
		\begin{itemize}
			\item B.W $40Mbps$
			\item Jitter $<50ms$ for HDTV quality
			\item Residual bit error rate $<10^6$ for HDTV quality
				stream using compressed format
			\item Loss/Error rate $<10^6$
		\end{itemize}
		\item MPEG-4 video streams:
		\begin{itemize}
			\item B.W $28.8-500kbps$
			\item Jitter $<150ms$ due to lower quality req.
			\item MPEG-4 has higher comp. rate, $\therefore$ less
				residual error
			\item Loss/Error rate $<10^5$
		\end{itemize}
	\end{itemize}
	\item Statement of level of quality actually achieved and delivered to
		customer
	\item should be same as offered QoS
	\begin{itemize}
		\item Determine what was actually achieved to asses level of
			performance achieved
	\end{itemize}
	\item Performance figures summarised for specified periods of time
\end{itemize}
\subsubsection{QoS Percieved by Customer}
\begin{itemize}
	\item Statement expressing level of quality exp.
	\item Perceived QoS - Degrees of Satisfaction, not in tech. terms
	\item Mean Opinion Score (MOS) specified by ITU-T Rec. P.800
\end{itemize}
\subsection{Service Level Management Functions}
\begin{itemize}
	\item SLM categorized into seven functions:
	\begin{itemize}
		\item SLA creation
		\begin{itemize}
			\item Create SLA template for specified services
		\end{itemize}
		\item SLA Negotiation
		\begin{itemize}
			\item Selecting applicable QoS params. in SLA
			\item Negotiating penalty in case of SLA violation
		\end{itemize}
		\item SLA Provisioning
		\begin{itemize}
			\item SP configure the network element/topology to
				provide service
		\end{itemize}
		\item SLA Monitorimg
		\begin{itemize}
			\item SP must verify degree of SLA assurance
			\item Perform surveillance on QoS parameter
				degredation/violation
		\end{itemize}
		\item SLA Maintenance
		\begin{itemize}
			\item  In case of QoS violation, analyses readon why
				degredation has occured, which params. degraded
			\item Notifies SLA provisioning to restore service
		\end{itemize}
		\item SLA Reporting
		\begin{itemize}
			\item Provides performance info to customers,
				periodically or on-demand
		\end{itemize}
		\item SLA Assessment
		\begin{itemize}
			\item Demands payments to customers
			\item Accommodates customers with penalty when violation
				occurs
		\end{itemize}
	\end{itemize}
	\item SLA provisioning and monitoring most important in net management
		layer
\end{itemize}
\subsubsection{SLA Monitoruing}
\begin{itemize}
	\item Input
	\begin{itemize}
		\item QoS Parameters
		\item SLA Contract
	\end{itemize}
	\item SLA Monitoring
	\item Output
	\begin{itemize}
		\item Problem Notification
		\item Performance
	\end{itemize}
\end{itemize}
\section{Network Performance Metrics}
\begin{itemize}
	\item Network Performance Metric (NPM)
	\begin{itemize}
		\item Basic metric of performance metric in net management layer
	\end{itemize}
	\item Four Types:
	\begin{itemize}
		\item Availability:
		\begin{itemize}
			\item $\%$ spec. t interval in which sys was available
				for normal use
			\item What is supposed to be available?
			\begin{itemize}
				\item Service, Host, Network
			\end{itemize}
			\item Reported as single monthly figure
			\begin{itemize}
				\item $99.99\%$ means service is unavailable for
					4 minutes during a month
			\end{itemize}
			\item Test by sending syutable packets, observing
				answering packets (latency, packet loss)
			\item Metrics:
			\begin{itemize}
				\item Connectivity: Physical connectivity of
					network elements
				\item Functionality: Whether associated sys
					works well or not
			\end{itemize}
		\end{itemize}
		\item Loss:
		\begin{itemize}
			\item Fraction of packets list in transit from host to
				another during specified t.
			\item Packet transport works on best-effort basis
			\item Moderate level of packet loss not in itself
				tolerable
			\begin{itemize}
				\item Some real-time services can tolerate some
					losses e.g. VoIP
				\item TCP resends lost packets at slower rate
			\end{itemize}
			\item Metrics:
			\begin{itemize}
				\item One way loss
				\item Round Trip (RT) Loss
			\end{itemize}
		\end{itemize}
		\item Delay:
		\begin{itemize}
			\item t taken for pkt to tracel from host to another
			\item $RT Delay = Forward Transport delay + Server delay +
				backward transport delay$
			\item Forward transport delay often $\neq$ backward
				transport dfelay
			\item Ping still most commonly used to measure latency
			\item Delay changes as conditions on net. vary
			\begin{itemize}
				\item e.g. Server load, traffic load, router
					load, routing function
			\end{itemize}
			\item For streaming, high delay/jitter (delay variation)
				can cause degredation on user-perceived QoS
			\item Metrics
			\begin{itemize}
				\item One Way delay
				\item RT Delay
				\item Jitter
			\end{itemize}
		\end{itemize}
		\item Utilization:
		\begin{itemize}
			\item Throughput for link expressed as $\%$ of access
				rate
			\item Throughput:
			\begin{itemize}
				\item Rate data is sent through net. (b/s,
					pkt/s, flows/s)
			\end{itemize}
			\item Metrics:
			\begin{itemize}
				\item Capacity
				\item B.W
				\item Throughput
			\end{itemize}
		\end{itemize}
	\end{itemize}
\end{itemize}
