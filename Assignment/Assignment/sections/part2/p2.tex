Each section of this assignment deals with three of the main performance
characteristics of a network, namely throughput, delay, and loss. These values
are affected by the number of users in the network, the distance between the
nodes in the network, and the bitrate. Question A deals with the differences in
bitrates, Question B with the differences in distances, and Question C Part 2
with the number of users in the network.

\subsection{Throughput}

For Question A Part 1, there is a throughput value of 1.6Kbps. This network has
a single user (sink) node, and a single access point (source) node. The bitrate
is determined in this case by the 0.05 second transmission interval.

\par Question A Part 2 investigates the throughput due to different bitrates.
With a bitrate of 1Mbps, there is a throughput of 1Mbps. A 1.5Mbps bitrate
corresponds to a 1.5Mbps throughput, a 5Mbps bitrate to a 5Mbps throughput, and
a 10Mbps bitrate to an approximately 10Mbps throughput. At a bitrate of 20Mbps
however, there is a throughput of only 13.7Mbps, due to the loss of packets.

\par Question B investigates the throughput due to different distances, using
the initial setup from Question A Part 1. At distances between 0 and 110 meters,
there is a throughput of 160Kbps. At 120 meters, due to lost packets, there is a
throughput of 5.2Kbps, and between 130 and 150 meters, no packets are received
at the sink node, giving a throughput of 0Kbps.

\par Question C Part 2 investigates the throughput with an increased number of
users within the network. For 1 to 10 sink nodes, there is no effect on the
throughput, giving a value of 1Mbps. With 20 and 50 sink nodes, there is a small
drop in throughput, with a value of 0.98992Mbps.

\begin{table}[H]
	\centering
	\caption{Throughput Values for QA Part 1 and 2, QB, and QC Part 2}
	\label{tab:P2TP}
	\begin{tabular}{|c|c|}
		\hline
		\multicolumn{2}{|c|}{Throughput (Kbps)} \\
		\hline
		QA Part 1 & 160 \\
		QA P2_{1Mbps} & 1000\\
		QA P2_{1.5Mbps} & 1500\\
		QA P2_{5Mbps} & 5000\\
		QA P2_{10Mbps} & 9999\\
		QA P2_{20Mbps} & 13747\\
		QB_{0-110m} & 160\\
		QB_{120} & 5.2\\
		QB_{130-150} & 0\\
		QC_{1-10Users} & 1000 \\
		QC_{20-50Users} & 989.92 \\
		\hline
	\end{tabular}
\end{table}

This table shows that attempting to increase the throughput by increasing the
bitrate will only work to a certain point, at which packet loss will begin to
occur, causing the throughput to drop below the defined bitrate.

\par The table also shows that distance between nodes has a much greater affect
on the throughput than the number of users within the network.

\subsection{Delay}

For Question A Part 1, the single source, single sink network has a delay
value of 0.49ms.

\par Question A Part 2 investigates the delay due to different bitrates.
With a bitrate of 1Mbps, there is a delay of 0.44ms. A 1.5Mbps bitrate
corresponds to a delay of 0.43ms, a 5Mbps bitrate to a delay of 0.438ms, and
a 10Mbps bitrate to a delay of 0.497ms. At a bitrate of 20Mbps
however, due to packet loss, there is a delay of 229ms.

\par Question B investigates the delay due to different distances, using
the initial setup from Question A Part 1. The delay steadily increases for
distances between 0 and 70 meters, from 0.27ms to 0.65ms. At 80 meters there
is a steeper increase, to a delay of 0.84ms. Following this there is again a
steady increase to a delay of 1.72ms at 110 meters, with a final spike to 9.3ms
visible at 120 meters. Following 120 meters there is no delay, as no packets are
being received.

\par Question C Part 2 investigates the delay with an increased number of
users within the network. For a single user, the delay is 0.44ms. With 5 users,
this delay increases to 2.8ms. With 10 users, the delay increases to 8.4ms. With
20 users there is a delay of 227.9ms, and finally, with 50 users there is a
delay of 1530.5ms.

\begin{table}[H]
	\centering
	\caption{Delay Values for QA Part 1 and 2, QB, and QC Part 2}
	\label{tab:P2Delay}
	\begin{tabular}{|c|c|}
		\hline
		\multicolumn{2}{|c|}{Delay (ms)} \\
		\hline
		QA Part 1 & 0.44 \\
		QA P2_{1Mbps} & 0.44\\
		QA P2_{1.5Mbps} & 0.43\\
		QA P2_{5Mbps} & 0.43\\
		QA P2_{10Mbps} & 0.49\\
		QA P2_{20Mbps} & 229\\
		QB & Refer to Equation 10 \\
		QB & MaxDelay = 9.36 \\
		QC_{1User} & 0.44\\
		QC_{5User} & 2.8\\
		QC_{10User} & 8.4\\
		QC_{20User} & 227.9\\
		QC_{50User} & 1530.5\\
		\hline
	\end{tabular}
\end{table}

This table shows that distance between nodes has a much less of an affect
on the delay than the number of users within the network. This is converse to
the throughput, which was affected more by the distance.

\par With 50 users, the delay is approximately 160 times greater than the
maximum delay within Question B in which the distance was being increased.

\subsection{Loss}

For Question A Part 1, the single source, single sink network has a Packet Loss
Ratio (PLR) of 0\% as no packets are lost.

\par Question A Part 2 investigates the PLR due to different bitrates.
No packet loss occurs for bitrates between 1 and 5Mbps, giving a PLR of 0\%. At
10Mbps there are 3 lost packets, giving a PLR of $1.2\time10^{-5}$\%. A bitrate
of 20Mbps corresponds to a 30\% PLR.

\par Question B investigates the loss due to different distances, using
the initial setup from Question A Part 1. For distances between 0 and 110 meters
no packet loss occurs giving a PLR of 0\%. At 120 meters, there is a PLR of
96.75\%, due to 387 of 400 packets being lost. Between 130 and 150 meters, no
packets were received, giving a PLR of 100\%.

\par Question C Part 2 investigates the loss with an increased number of
users within the network. For 1, 5, and 10 Sinks, there are no lost packets,
giving a PLR of 0\%. For 20 and 50 sinks, there is a packet loss ratio of
1.08\%.

\begin{table}[H]
	\centering
	\caption{Packet Loss Ratio Values for QA Part 1 and 2, QB, and QC Part 2}
	\label{tab:P2PLR}
	\begin{tabular}{|c|c|}
		\hline
		\multicolumn{2}{|c|}{Packet Loss (\%)} \\
		\hline
		QA Part 1 & 0 \\
		QA P2_{1Mbps} & 0\\
		QA P2_{1.5Mbps} & 0\\
		QA P2_{5Mbps} & 0\\
		QA P2_{10Mbps} & $1.2\times10^{-5}$\\
		QA P2_{20Mbps} & 30.7\\
		QB_{0-110m} & 0\\
		QB_{120} & 96.75\\
		QB_{130-150} & 100\\
		QC_{1-10Users} & 0 \\
		QC_{20-50Users} & 1.08 \\
		\hline
	\end{tabular}
\end{table}

Similarly to Table \ref{tab:P2TP}, the loss is affected much more by the
distance between the nodes than by the number of nodes. Much higher loss is
experienced at larger distances than with larger numbers of nodes in the
network.

\subsection{Conclusion}

From the previous comparisons, it can be seen that both throughput and loss are
affected more by distance than by the numbers of users in the system, whereas
delay is affected much more by the number of users than by the distance to the
source node.

As such, if throughput and loss mitigation are more important characteristics in
the network, then users (sink nodes) should be kept within close proximity of
the source node, at a maximum distance of 120 meters. If delay is the more
important characteristic, then the number of sink nodes should be limited, as
over 10 users tends to dramatically increase the delay within the system.
